\documentclass{TD}
\iutset{%
  cours={LP Gestion de projet Web},%
  annee={2013--2014},%
  titre={Développement Dirigé par les Tests},%
  numero={2}}
\usepackage{hyperref}
\usepackage{banner}
\begin{document}
Dans ce TD, vous allez développer, avec \texttt{playframework}, un
convertisseur en ligne pour traduire des entiers entre leur représentation en
décimal et leur représentation en chiffres romains.

Le véritable but de cette session est de s'initier à la méthode agile appelée
\emph{Test-Driven Development} ou TDD, c'est à dire le dévelopement dirigé par
les tests.  On utilisera également une autre méthode agile appelée \emph{Pair
  Programming} ou l'on travaille en binôme.  Mais, pour commencer, le
responsable de TD animera ce qu'on appelle une session de \emph{Randori}.

\exercice{Github}

Pour vous aider à démarrer cette application, un squelette déjà fonctionnel
est disponible sur github.
\begin{compactitem}
\item Vous devez vous créer un compte sur github su vous n'en avez pas déjà un
\item Loggez vous sur votre compte github
\item Visitez le projet \texttt{tdd-chiffres-romains} à l'url:
\begin{center}
  \url{https://github.com/denys-duchier/tdd-chiffres-romains}
\end{center}
\item Cliquez \texttt{Fork} pour cloner ce projet sur votre compte github
\end{compactitem}
Vous pouvez maintenant obtenir ce code sur votre compte à l'IUT.  Placez-vous
par exemple dans votre home:
\begin{bash}
cd
\end{bash}
puis clonez votre copie du projet:
\begin{bash}
git clone https://....
\end{bash}
où l'URL est celui donné dans la page web de votre copie du projet (la boite
intitulée ``\texttt{HTTPS clone URL}'' dans la marge droite de la page).
Entrez dans le projet et observez:
\begin{bash}
cd tdd-chiffres-romains
git remote -v
git branch -av
\end{bash}

\exercice{Mise en place des binômes}

Puisque vous allez travailler en \emph{pair programming}, il va falloir mettre
en place la possibilité de faire des merges entre vos dépôts respectifs.
\begin{compactitem}
\item Choisissez un binôme
\item Je désignerai par A et B les deux membres du binôme
\end{compactitem}

\banner{A travaille / B regarde}

A ajoute son nom au fichier \texttt{README}, puis fait:
\begin{bash}
git commit -a -m "ajout de A au README"
git push
\end{bash}

\banner{A regarde / B travaille}

B veut pouvoir récupérer les mises à jours de A.  Pour cela il doit indiquer à
git où se trouve le dépôt de A.  Cela se fait en ajoutant un \emph{remote}:
\begin{bash}
git remote add binome https://...
git remote -v
\end{bash}
où l'url est celui du dépôt de A.  Il faut maintenant télécharger l'historique
de A:
\begin{bash}
git fetch binome
\end{bash}
Observez:
\begin{bash}
git branch -av
\end{bash}
Vous avez maintenant des \emph{remote tracking branches} pour la/les branches
du dépôt de A.  Mais votre branche \texttt{master} n'a pas encore été modifiée:
il reste à faire le merge:
\begin{bash}
git merge binome/master
git branch -av
git log
\end{bash}
Vous ajoutez votre nom au \texttt{README}, puis:
\begin{bash}
git commit -a -m "ajout de B au README"
git push
\end{bash}

\banner{A travaille / B regarde}

A veut pouvoir récupérer les mises à jours de B.  Donc à son tour:
\begin{bash}
git remote add binome https://...
git remote -v
\end{bash}
où l'url est celui du dépôt de B.
\begin{bash}
git fetch binome
\end{bash}
Observez:
\begin{bash}
git branch -av
\end{bash}
Puis faites le merge:
\begin{bash}
git merge binome/master
git branch -av
git log
\end{bash}

\exercice{Vérification du fonctionnement}

Vérifiez que l'application se lance:
\begin{bash}
play run
\end{bash}
et qu'elle est accessible sur \url{http://localhost:9000}.

Vérifiez également que le framework de test est en place:
\begin{bash}
play test
\end{bash}

\exercice{Randori}

Le responsable de TD va animer une session de Randori, c'est à dire où 2
étudiants sont au poste video-projeté: l'un observe et conseille, l'autre
programme et décrit à haute voix ce qu'il fait. Après 1 ou 2 itération,
l'étudiant programmeur retourne à sa place, l'observateur le remplace au
clavier et un autre étudiant prend la place de l'observateur.

\exercice{Pair Programming}

Après un amorçage suffisant en Randori, les binômes continuerons par eux-même:
l'un joue le rôle d'observateur, l'autre de programmeur; puis, après 1 ou 2
itérations, ils échangent leurs rôles, mais gardent leurs places respectives:
les commits de l'un sont publiés par \texttt{push} puis récupérés par l'autre
par \texttt{pull}.

\end{document}
